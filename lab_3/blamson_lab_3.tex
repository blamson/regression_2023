% Options for packages loaded elsewhere
\PassOptionsToPackage{unicode}{hyperref}
\PassOptionsToPackage{hyphens}{url}
\PassOptionsToPackage{dvipsnames,svgnames,x11names}{xcolor}
%
\documentclass[
  letterpaper,
  DIV=11,
  numbers=noendperiod]{scrartcl}

\usepackage{amsmath,amssymb}
\usepackage{lmodern}
\usepackage{iftex}
\ifPDFTeX
  \usepackage[T1]{fontenc}
  \usepackage[utf8]{inputenc}
  \usepackage{textcomp} % provide euro and other symbols
\else % if luatex or xetex
  \usepackage{unicode-math}
  \defaultfontfeatures{Scale=MatchLowercase}
  \defaultfontfeatures[\rmfamily]{Ligatures=TeX,Scale=1}
\fi
% Use upquote if available, for straight quotes in verbatim environments
\IfFileExists{upquote.sty}{\usepackage{upquote}}{}
\IfFileExists{microtype.sty}{% use microtype if available
  \usepackage[]{microtype}
  \UseMicrotypeSet[protrusion]{basicmath} % disable protrusion for tt fonts
}{}
\makeatletter
\@ifundefined{KOMAClassName}{% if non-KOMA class
  \IfFileExists{parskip.sty}{%
    \usepackage{parskip}
  }{% else
    \setlength{\parindent}{0pt}
    \setlength{\parskip}{6pt plus 2pt minus 1pt}}
}{% if KOMA class
  \KOMAoptions{parskip=half}}
\makeatother
\usepackage{xcolor}
\setlength{\emergencystretch}{3em} % prevent overfull lines
\setcounter{secnumdepth}{-\maxdimen} % remove section numbering
% Make \paragraph and \subparagraph free-standing
\ifx\paragraph\undefined\else
  \let\oldparagraph\paragraph
  \renewcommand{\paragraph}[1]{\oldparagraph{#1}\mbox{}}
\fi
\ifx\subparagraph\undefined\else
  \let\oldsubparagraph\subparagraph
  \renewcommand{\subparagraph}[1]{\oldsubparagraph{#1}\mbox{}}
\fi

\usepackage{color}
\usepackage{fancyvrb}
\newcommand{\VerbBar}{|}
\newcommand{\VERB}{\Verb[commandchars=\\\{\}]}
\DefineVerbatimEnvironment{Highlighting}{Verbatim}{commandchars=\\\{\}}
% Add ',fontsize=\small' for more characters per line
\usepackage{framed}
\definecolor{shadecolor}{RGB}{241,243,245}
\newenvironment{Shaded}{\begin{snugshade}}{\end{snugshade}}
\newcommand{\AlertTok}[1]{\textcolor[rgb]{0.68,0.00,0.00}{#1}}
\newcommand{\AnnotationTok}[1]{\textcolor[rgb]{0.37,0.37,0.37}{#1}}
\newcommand{\AttributeTok}[1]{\textcolor[rgb]{0.40,0.45,0.13}{#1}}
\newcommand{\BaseNTok}[1]{\textcolor[rgb]{0.68,0.00,0.00}{#1}}
\newcommand{\BuiltInTok}[1]{\textcolor[rgb]{0.00,0.23,0.31}{#1}}
\newcommand{\CharTok}[1]{\textcolor[rgb]{0.13,0.47,0.30}{#1}}
\newcommand{\CommentTok}[1]{\textcolor[rgb]{0.37,0.37,0.37}{#1}}
\newcommand{\CommentVarTok}[1]{\textcolor[rgb]{0.37,0.37,0.37}{\textit{#1}}}
\newcommand{\ConstantTok}[1]{\textcolor[rgb]{0.56,0.35,0.01}{#1}}
\newcommand{\ControlFlowTok}[1]{\textcolor[rgb]{0.00,0.23,0.31}{#1}}
\newcommand{\DataTypeTok}[1]{\textcolor[rgb]{0.68,0.00,0.00}{#1}}
\newcommand{\DecValTok}[1]{\textcolor[rgb]{0.68,0.00,0.00}{#1}}
\newcommand{\DocumentationTok}[1]{\textcolor[rgb]{0.37,0.37,0.37}{\textit{#1}}}
\newcommand{\ErrorTok}[1]{\textcolor[rgb]{0.68,0.00,0.00}{#1}}
\newcommand{\ExtensionTok}[1]{\textcolor[rgb]{0.00,0.23,0.31}{#1}}
\newcommand{\FloatTok}[1]{\textcolor[rgb]{0.68,0.00,0.00}{#1}}
\newcommand{\FunctionTok}[1]{\textcolor[rgb]{0.28,0.35,0.67}{#1}}
\newcommand{\ImportTok}[1]{\textcolor[rgb]{0.00,0.46,0.62}{#1}}
\newcommand{\InformationTok}[1]{\textcolor[rgb]{0.37,0.37,0.37}{#1}}
\newcommand{\KeywordTok}[1]{\textcolor[rgb]{0.00,0.23,0.31}{#1}}
\newcommand{\NormalTok}[1]{\textcolor[rgb]{0.00,0.23,0.31}{#1}}
\newcommand{\OperatorTok}[1]{\textcolor[rgb]{0.37,0.37,0.37}{#1}}
\newcommand{\OtherTok}[1]{\textcolor[rgb]{0.00,0.23,0.31}{#1}}
\newcommand{\PreprocessorTok}[1]{\textcolor[rgb]{0.68,0.00,0.00}{#1}}
\newcommand{\RegionMarkerTok}[1]{\textcolor[rgb]{0.00,0.23,0.31}{#1}}
\newcommand{\SpecialCharTok}[1]{\textcolor[rgb]{0.37,0.37,0.37}{#1}}
\newcommand{\SpecialStringTok}[1]{\textcolor[rgb]{0.13,0.47,0.30}{#1}}
\newcommand{\StringTok}[1]{\textcolor[rgb]{0.13,0.47,0.30}{#1}}
\newcommand{\VariableTok}[1]{\textcolor[rgb]{0.07,0.07,0.07}{#1}}
\newcommand{\VerbatimStringTok}[1]{\textcolor[rgb]{0.13,0.47,0.30}{#1}}
\newcommand{\WarningTok}[1]{\textcolor[rgb]{0.37,0.37,0.37}{\textit{#1}}}

\providecommand{\tightlist}{%
  \setlength{\itemsep}{0pt}\setlength{\parskip}{0pt}}\usepackage{longtable,booktabs,array}
\usepackage{calc} % for calculating minipage widths
% Correct order of tables after \paragraph or \subparagraph
\usepackage{etoolbox}
\makeatletter
\patchcmd\longtable{\par}{\if@noskipsec\mbox{}\fi\par}{}{}
\makeatother
% Allow footnotes in longtable head/foot
\IfFileExists{footnotehyper.sty}{\usepackage{footnotehyper}}{\usepackage{footnote}}
\makesavenoteenv{longtable}
\usepackage{graphicx}
\makeatletter
\def\maxwidth{\ifdim\Gin@nat@width>\linewidth\linewidth\else\Gin@nat@width\fi}
\def\maxheight{\ifdim\Gin@nat@height>\textheight\textheight\else\Gin@nat@height\fi}
\makeatother
% Scale images if necessary, so that they will not overflow the page
% margins by default, and it is still possible to overwrite the defaults
% using explicit options in \includegraphics[width, height, ...]{}
\setkeys{Gin}{width=\maxwidth,height=\maxheight,keepaspectratio}
% Set default figure placement to htbp
\makeatletter
\def\fps@figure{htbp}
\makeatother

\KOMAoption{captions}{tableheading}
\makeatletter
\makeatother
\makeatletter
\makeatother
\makeatletter
\@ifpackageloaded{caption}{}{\usepackage{caption}}
\AtBeginDocument{%
\ifdefined\contentsname
  \renewcommand*\contentsname{Table of contents}
\else
  \newcommand\contentsname{Table of contents}
\fi
\ifdefined\listfigurename
  \renewcommand*\listfigurename{List of Figures}
\else
  \newcommand\listfigurename{List of Figures}
\fi
\ifdefined\listtablename
  \renewcommand*\listtablename{List of Tables}
\else
  \newcommand\listtablename{List of Tables}
\fi
\ifdefined\figurename
  \renewcommand*\figurename{Figure}
\else
  \newcommand\figurename{Figure}
\fi
\ifdefined\tablename
  \renewcommand*\tablename{Table}
\else
  \newcommand\tablename{Table}
\fi
}
\@ifpackageloaded{float}{}{\usepackage{float}}
\floatstyle{ruled}
\@ifundefined{c@chapter}{\newfloat{codelisting}{h}{lop}}{\newfloat{codelisting}{h}{lop}[chapter]}
\floatname{codelisting}{Listing}
\newcommand*\listoflistings{\listof{codelisting}{List of Listings}}
\makeatother
\makeatletter
\@ifpackageloaded{caption}{}{\usepackage{caption}}
\@ifpackageloaded{subcaption}{}{\usepackage{subcaption}}
\makeatother
\makeatletter
\@ifpackageloaded{tcolorbox}{}{\usepackage[many]{tcolorbox}}
\makeatother
\makeatletter
\@ifundefined{shadecolor}{\definecolor{shadecolor}{rgb}{.97, .97, .97}}
\makeatother
\makeatletter
\makeatother
\ifLuaTeX
  \usepackage{selnolig}  % disable illegal ligatures
\fi
\IfFileExists{bookmark.sty}{\usepackage{bookmark}}{\usepackage{hyperref}}
\IfFileExists{xurl.sty}{\usepackage{xurl}}{} % add URL line breaks if available
\urlstyle{same} % disable monospaced font for URLs
\hypersetup{
  pdftitle={Lab 3},
  pdfauthor={Brady Lamson},
  colorlinks=true,
  linkcolor={blue},
  filecolor={Maroon},
  citecolor={Blue},
  urlcolor={Blue},
  pdfcreator={LaTeX via pandoc}}

\title{Lab 3}
\author{Brady Lamson}
\date{}

\begin{document}
\maketitle
\ifdefined\Shaded\renewenvironment{Shaded}{\begin{tcolorbox}[frame hidden, borderline west={3pt}{0pt}{shadecolor}, interior hidden, breakable, enhanced, sharp corners, boxrule=0pt]}{\end{tcolorbox}}\fi

\begin{Shaded}
\begin{Highlighting}[]
\NormalTok{data }\OtherTok{\textless{}{-}} \FunctionTok{read.table}\NormalTok{(}\AttributeTok{header =}\NormalTok{ T, }\AttributeTok{file =} \StringTok{"../datasets/properties.txt"}\NormalTok{)}
\end{Highlighting}
\end{Shaded}

\hypertarget{section}{%
\section{1.1}\label{section}}

\hypertarget{section-1}{%
\subsection{3.}\label{section-1}}

\begin{Shaded}
\begin{Highlighting}[]
\NormalTok{data\_reg }\OtherTok{\textless{}{-}} \FunctionTok{lm}\NormalTok{(Rent.Rate }\SpecialCharTok{\textasciitilde{}} \SpecialCharTok{{-}}\DecValTok{1} \SpecialCharTok{+}\NormalTok{ Op.Expense, }\AttributeTok{data =}\NormalTok{ data)}
\NormalTok{data\_reg }\SpecialCharTok{|\textgreater{}} \FunctionTok{summary}\NormalTok{()}
\end{Highlighting}
\end{Shaded}

\begin{verbatim}

Call:
lm(formula = Rent.Rate ~ -1 + Op.Expense, data = data)

Residuals:
    Min      1Q  Median      3Q     Max 
-4.6109 -1.3563 -0.2843  2.1746  9.1239 

Coefficients:
           Estimate Std. Error t value Pr(>|t|)    
Op.Expense  1.47817    0.03963    37.3   <2e-16 ***
---
Signif. codes:  0 '***' 0.001 '**' 0.01 '*' 0.05 '.' 0.1 ' ' 1

Residual standard error: 3.575 on 80 degrees of freedom
Multiple R-squared:  0.9456,    Adjusted R-squared:  0.9449 
F-statistic:  1391 on 1 and 80 DF,  p-value: < 2.2e-16
\end{verbatim}

\[
Y_i = 1.478X_i + \epsilon_i
\]

\hypertarget{section-2}{%
\subsection{4.}\label{section-2}}

Forcing the line to go through the origin does \textbf{not} seem
appropriate for this data set.

\hypertarget{section-3}{%
\subsection{5.}\label{section-3}}

The pattern seems relatively flat, maybe a slight positive relationship
as theres a larger clustering around the top on the righthand side than
the left.

\hypertarget{section-4}{%
\subsection{6.}\label{section-4}}

The residuals do not sum to zero, they sum to 66.27

\hypertarget{part-b}{%
\section{Part B}\label{part-b}}

\hypertarget{section-5}{%
\subsection{2.}\label{section-5}}

\begin{Shaded}
\begin{Highlighting}[]
\NormalTok{standard\_reg }\OtherTok{\textless{}{-}} \FunctionTok{lm}\NormalTok{(Rent.Rate }\SpecialCharTok{\textasciitilde{}}\NormalTok{ Op.Expense, }\AttributeTok{data =}\NormalTok{ data)}
\NormalTok{X }\OtherTok{\textless{}{-}} \FunctionTok{model.matrix}\NormalTok{(standard\_reg)}

\FunctionTok{solve}\NormalTok{(}\FunctionTok{t}\NormalTok{(X) }\SpecialCharTok{\%*\%}\NormalTok{ X)}
\end{Highlighting}
\end{Shaded}

\begin{verbatim}
            (Intercept)   Op.Expense
(Intercept)  0.18817287 -0.018148689
Op.Expense  -0.01814869  0.001873288
\end{verbatim}

\hypertarget{section-6}{%
\subsection{4.}\label{section-6}}

\begin{Shaded}
\begin{Highlighting}[]
\NormalTok{Y }\OtherTok{\textless{}{-}}\NormalTok{ data}\SpecialCharTok{$}\NormalTok{Rent.Rate}
\NormalTok{b }\OtherTok{\textless{}{-}} \FunctionTok{solve}\NormalTok{(}\FunctionTok{t}\NormalTok{(X) }\SpecialCharTok{\%*\%}\NormalTok{ X) }\SpecialCharTok{\%*\%}\NormalTok{ (}\FunctionTok{t}\NormalTok{(X) }\SpecialCharTok{\%*\%}\NormalTok{ Y)}
\NormalTok{b}
\end{Highlighting}
\end{Shaded}

\begin{verbatim}
                  [,1]
(Intercept) 12.4702588
Op.Expense   0.2754531
\end{verbatim}

The coefficients in \(\vec{b}\) are the same as the output of summary.

\hypertarget{section-7}{%
\subsection{5.}\label{section-7}}

\begin{Shaded}
\begin{Highlighting}[]
\NormalTok{y\_hat }\OtherTok{\textless{}{-}}\NormalTok{ X }\SpecialCharTok{\%*\%}\NormalTok{ b}
\NormalTok{y\_hat }\SpecialCharTok{|\textgreater{}} \FunctionTok{head}\NormalTok{()}
\end{Highlighting}
\end{Shaded}

\begin{verbatim}
      [,1]
1 13.85303
2 14.72622
3 13.29662
4 15.41761
5 14.94107
6 15.07329
\end{verbatim}

These values are the same as the fitted values from the regression.

\hypertarget{section-8}{%
\subsection{6.}\label{section-8}}

\begin{Shaded}
\begin{Highlighting}[]
\NormalTok{(Y }\SpecialCharTok{{-}}\NormalTok{ y\_hat) }\SpecialCharTok{|\textgreater{}} \FunctionTok{head}\NormalTok{()}
\end{Highlighting}
\end{Shaded}

\begin{verbatim}
        [,1]
1 -0.3530332
2 -2.7262194
3 -2.7966180
4 -0.4176066
5 -0.9410728
6 -4.5732903
\end{verbatim}

These match the residuals of the regression.

\hypertarget{section-9}{%
\subsection{7.}\label{section-9}}

\begin{Shaded}
\begin{Highlighting}[]
\FunctionTok{sqrt}\NormalTok{(}\FloatTok{1.575}\SpecialCharTok{\^{}}\DecValTok{2} \SpecialCharTok{*} \FunctionTok{solve}\NormalTok{(}\FunctionTok{t}\NormalTok{(X) }\SpecialCharTok{\%*\%}\NormalTok{ X))}
\end{Highlighting}
\end{Shaded}

\begin{verbatim}
Warning in sqrt(1.575^2 * solve(t(X) %*% X)): NaNs produced
\end{verbatim}

\begin{verbatim}
            (Intercept) Op.Expense
(Intercept)   0.6832176        NaN
Op.Expense          NaN 0.06816835
\end{verbatim}

The diagonals are the same as the std. error from the output of summary.

\hypertarget{part-c}{%
\section{Part C}\label{part-c}}

\begin{Shaded}
\begin{Highlighting}[]
\NormalTok{length }\OtherTok{\textless{}{-}} \FunctionTok{c}\NormalTok{(}\DecValTok{63}\NormalTok{, }\DecValTok{63}\NormalTok{, }\DecValTok{63}\NormalTok{, }\DecValTok{63}\NormalTok{, }\DecValTok{63}\NormalTok{, }\DecValTok{63}\NormalTok{, }\DecValTok{63}\NormalTok{, }\DecValTok{63}\NormalTok{, }\DecValTok{63}\NormalTok{)}
\NormalTok{weight}\OtherTok{\textless{}{-}} \FunctionTok{c}\NormalTok{(}\DecValTok{136}\NormalTok{, }\DecValTok{198}\NormalTok{, }\DecValTok{194}\NormalTok{, }\DecValTok{140}\NormalTok{, }\DecValTok{93}\NormalTok{, }\DecValTok{172}\NormalTok{, }\DecValTok{116}\NormalTok{, }\DecValTok{174}\NormalTok{, }\DecValTok{145}\NormalTok{)}
\NormalTok{snakes }\OtherTok{\textless{}{-}} \FunctionTok{data.frame}\NormalTok{(}\AttributeTok{Length =}\NormalTok{ length, }\AttributeTok{Weight =}\NormalTok{ weight)}

\NormalTok{my.reg }\OtherTok{\textless{}{-}} \FunctionTok{lm}\NormalTok{(Weight }\SpecialCharTok{\textasciitilde{}}\NormalTok{ Length, }\AttributeTok{data =}\NormalTok{ snakes)}
\end{Highlighting}
\end{Shaded}

There's only 1 unique value for length, so that ruins everything here.



\end{document}
